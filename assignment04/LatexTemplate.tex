\documentclass[12pt]{amsart}

\usepackage{amsmath}
\usepackage{amssymb}
\usepackage{amsthm}
\usepackage{mathrsfs}
\usepackage{enumerate}

%All of these let you just type $\N$ for the $\N$atural numbers symbol and so on
\newcommand{\N}{\mathbb{N}}
\newcommand{\C}{\mathbb{C}}
\newcommand{\R}{\mathbb{R}}
\newcommand{\Z}{\mathbb{Z}}
\newcommand{\Q}{\mathbb{Q}}

\setlength{\parindent}{36pt}
\setlength{\parskip}{0em}

% \begin{align} A=B & Axiom 1    Puts equation on left and axiom on right

\begin{document}

\title{Homework 4}
\date{February 13, 2017}
\author{Leandro Ribeiro\\(Worked with Kyle Franke and Joyce Gomez)}

\maketitle

\newtheorem*{prop2.18}{Proposition 2.18}
\begin{prop2.18}
	(i) For all k $\in$ $\N$, $k^3$ + 2k is divisible by 3.
	\\\indent(ii) For all k $\in$ $\N$, $k^4$ - 6$k^3$ + 11$k^2$ - 6k is divisible by 4.
	\\\indent(iii) For all k $\in$ $\N$, $k^3$ + 5k is divisible by 6.
\end{prop2.18}

\begin{proof}
	(i) Let's first identify P(n). We put P(n) to be the sentence, "$n^3$ + 2n is divisible by 3." Let us first consider P(1). 
\\\textbf{Base.} P(1) asserts $1^3$ + 2 = 3, we conclude that P(1) holds.
\\\textbf{Successor} We have that P(n) holds. That is, 3 divides $n^3$ +2n. Consider $(n + 1)^3$ + 2(n + 1). $(n +1)^3$ + 2n + 2 = \dots = $n^3$ + 3$n^2$ + 2n + 1 = $n^3$ + 2n + 3$n^2$ + 1. Now our inductive hypothesis tells us that 3 divides $n^3$ + 2n = 3 $\cdot$ j. We now see that $(n + 1)^3$ + 2(n + 1) = 3j + 3($n^2$ + 1) = 3(j + $n^2$ + 1). Since j + $n^2$ + 1 is an integer, we conclude that 3 divides (n + 1$)^3$ + 2(n + 1). That is to say, P(n + 1) holds. By the principle of induction the proposition holds.
\\\indent (ii) Let's first identify P(n). We put P(n) to be the sentence, "$n^4$ - 6$n^3$ + 11$n^2$ - 6n is divisible by 4". Let's consider P(1).
\\\textbf{Base.} P(1) asserts 1 - 6 + 11 - 6 = 0. Proposition 1.17 tells us 0 is divisible by any integer, so we conclude that P(1) holds.
\\\textbf{Successor.} Suppose P(n) holds. That is, $n^4$ - 6$n^3$ + 11$n^2$ - 6n is divisible by 4. Consider $(n + 1)^4$ - 6$(n + 1)^3$ + 11$(n + 1)^2$ - 6(n + 1). $(n + 1)^4$ - 6$(n + 1)^3$ + 11$(n + 1)^2$ - 6(n + 1) = $(n + 1)^4$ - 6$(n + 1)^3$ + 11$(n + 1)^2$ - 6n + 6 = \dots = ($n^4$ - 6$n^3$ + 11$n^2$ - 6n) + 4$n^3$ - 12$n^2$ + 8n. Now our inductive hypothesis tells us that 4 divides $n^4$ - 6$n^3$ + 11$n^2$ - 6n = 4 $\cdot$ j. We now see that ($n^4$ - 6$n^3$ + 11$n^2$ - 6n) + 4$n^3$ - 12$n^2$ + 8n = 4j + 4$n^3$ - 12$n^2$ + 8n = 4j + 4($n^3$ - 3$n^2$ + 2n). Letting $n^3$ - 3$n^2$ + 2n = i, we have 4j + 4($n^3$ - 3$n^2$ + 2n) = 4j + 4i = 4(j + i). \\Thus 4 divides $(n + 1)^4$ - 6$(n + 1)^3$ + 11$(n + 1)^2$ - 6(n + 1), verifying that P(n + 1) holds. By the principle of induction we conclude $n^4$ - 6$n^3$ + 11$n^2$ - 6n is divisible by 4.
\\\indent (iii) Let's first identify P(n). We put P(n) to be the sentence "$n^3$ + 5n is divisible by 6". Let's first consider P(1).
\\\textbf{Base.} P(1) asserts $1^3$ + 5 is divisible by 6. Since $1^3$ + 5 = 6, we conclude that P(1) holds.
\\\textbf{successor} Assume P(n) holds. That is, $n^3$ + 5n is divisible by 6. Consider $(n + 1)^3$ + 5(n + 1). $(n + 1)^3$ + 5n + 5 = $n^3$ + 3$n^2$ + 3n + 1 + 5n + 5 = ($n^3$ + 5n) + ($3n^2$ + 3n + 6) = ($n^3$ + 5n) + 3($n^2$ + n + 2). In order to move forward, we first need to prove $n^2$ + n + 2 is divisible by 2 (proven below). Now that we know $n^2$ + n + 2 is even, we have 2 $\cdot$ j = $n^2$ + n + 2 for some j. We thus have that 3($n^2$ + n + 2) = 3 $\cdot$ 2 $\cdot$ j = 6 $\cdot$ j. Letting $n^3$ + 5n = 6 $\cdot$ i, we see that $(n + 1)^3$ + 5n = 6i + 6j = 6(i + j). Thus 6 divides $(n + 1)^3$ + 5(n + 1), verifying that P(n + 1) holds. By the principle of induction we conclude $n^3$ + 5n is divisible by 6.
\end{proof}

\newtheorem*{propExtra}{Proposition}
\begin{propExtra}
	For all n $\in$ $\N$, $n^2$ + n + 2 is divisible by 2.
\end{propExtra}

\begin{proof}
	Let's first identify P(n). We put P(n) to be the sentence "$n^2$ + n + 2 is divisible by 2". Let's first consider P(1).
	\\\textbf{Base.} P(1) asserts $1^2$ + 1 + 2 is divisible by 2. Since $1^2$ + 1 + 2 = 4, and 4 is divisible by 2, we conclude that P(1) holds.
	\\\textbf{Successor.} Assume P(n) holds. That is $n^2$ + n + 2 is divisible by 2. Consider $(n + 1)^2$ + n + 1 + 2. $(n + 1)^2$ + n + 3 = $n^3$ + 2n + 1 + n + 3 = $n^3$ + 3n + 4. Rewriting we have ($n^2$ + n + 2) + (2n + 2). Since P(n) holds, $n^2$ + n + 2 = 2j for some j. We have 2(j + n + 1). We conclude that $(n + 1)^2$ + n + 1 + 2 is divisible by 2, verifying that P(n + 1) holds. By the principle of induction the proposition holds.
\end{proof}


\newtheorem*{prop2.21}{Proposition 2.21}
\begin{prop2.21}
There exists no integer x such that 0 $<$ x $<$ 1.
\end{prop2.21}

\begin{proof}
	Suppose toward a contradiction there exists an integer x such that 0 $<$ x $<$ 1. By definition, x - 0  $\in$ $\N$ and by our axioms for the integers x $\in$ $\N$. x $<$ 1, but by Proposition 2.20 (proven below), x $\geq$ 1 which is absurd.
\end{proof}

\newtheorem*{prop2.20}{Proposition 2.20}
\begin{prop2.20}
	For all k $\in$ $\N$, k $\geq$ 1.
\end{prop2.20}

\begin{proof}
Let's formulate P(n) . "n $\geq$ 1"
\\\textbf{Base.} For this case, 1 = 1, so 1 $\geq$  1. Hence P(1) holds.
\\\textbf{Successor.} Suppose P(n) holds. That is 1 $\leq$ n. Consider n + 1 - n. Commuting we see that 1 + n - n = 1. Therefore n + 1 - n $\in$ $\N$. We deduce that n $<$ n + 1 since P(n) holds 1 $\leq$ n. The relation $\leq$ is transitive, so 1 $\leq$ n $\leq$ n + 1 implies 1 $\leq$ n + 1. We have proven the successor case by the principle of mathematical induction, we are done.
\end{proof}

\newtheorem*{prop2.23}{Proposition 2.23}
\begin{prop2.23}
	Let m, n $\in$  $\N$. If n is divisible by m then m $\leq$ n.
\end{prop2.23}

\begin{proof}
	We argue by induction on the claim P(n) "m $\leq$ n."
	\\\textbf{Base.} n = 1. In this case 1 = j $\cdot$ m for some j $\in$ $\Z$ by definition of divisibilty. j $\in$ $\N$ by proposition 2.11. m = j = n = 1, Thus P(1) holds.
	\\\textbf{Successor.} Suppose P(n) holds. That is, m $\leq$ n. By definition, this means n - m $\in$ $\N$ or n - m  = 0. If n - m $\in$ $\N$, n - m + 1 $\in$ $\N$ because $\N$ is closed under addition, and 1 $\in$ $\N$ by proposition 2.14(i). Otherwise, if n - m = 0, then n - m + 1 = 0 + 1 = 1 $\in$ $\N$ by proposition 2.14.(i). We have proven the successor case by the induction.
\end{proof}

\newtheorem*{prop2.24}{Proposition 2.24}
\begin{prop2.24}
	For all k $\in$ $\N$, $k^2$ + 1 $>$ k.
\end{prop2.24}

\begin{proof}
	We argue by induction the claim P(n) "$n^2$ + 1 $>$ n." Let's first observe P(1)
	\\\textbf{Base.} n = 1. In this case $1^2$ + 1 = 2 $>$ 1. Thus P(1) holds.
	\\\textbf{Successor.} Suppose P(n) holds. That is, $n^2$ + 1 $>$ n. Consider \\$(n + 1)^2$ + 1. $(n + 1)^2$ + 1 = $n^2$ + 2n + 2 = $n^2$ + 2 + 2n = $n^2$ + 1 + 1 + 2n $>$ n + 1 + 2n. n + 1 + 2n $>$ n + 1 by proposition 2.7(i), since 2n $>$ 0. By transitivity of $>$, $(n + 1)^2$  + 1 $>$ (n + 1). By induction, we've proven the proposition.
\end{proof}

\newtheorem*{prop2.27}{Proposition 2.27}
\begin{prop2.27}
	For all integers k $\geq$ 2, $k^2$ $<$ $k^3$.
\end{prop2.27}

\begin{proof}
	We argue by induction the claim P(n) "$n^2$ $<$ $n^3$." Let's first observe P(2)
	\\\textbf{Base.} n = 2. In this case, $2^2$ $<$ $2^3$. $2^2$ = 4 $<$ $2^3$ = 8. Thus P(2) holds.
	\\\textbf{Successor} Suppose P(n) holds. That is, $n^2$ $<$ $n^3$. Consider $(n + 1)^3$. $(n + 1)^3$ = $n^3$ + 3$n^2$ + 3n + 1 $>$ $n^2$ + 3$n^2$ + 3n + 1 = 4$n^2$ + 3n + 1 = 3$n^2$ + n + $n^2$ + 2n + 1. 3$n^2$ + n + $n^2$ + 2n + 1 $>$ $n^2$ + 2n + 1 by proposition 2.7(i), since 3$n^2$ + n $>$ 0. $n^2$ + 2n + 1 = $(n + 1)^2$, so by transitivity of $>$, $(n + 1)^2$ $<$ $(n + 1)^3$. We have thus proven this proposition by induction.
\end{proof}
\end{document}
