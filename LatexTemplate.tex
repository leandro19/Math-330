\documentclass[12pt]{amsart}

\usepackage{amsmath}
\usepackage{amssymb}
\usepackage{amsthm}
\usepackage{mathrsfs}
\usepackage{enumerate}

%All of these let you just type $\N$ for the $\N$atural numbers symbol and so on
\newcommand{\N}{\mathbb{N}}
\newcommand{\C}{\mathbb{C}}
\newcommand{\R}{\mathbb{R}}
\newcommand{\Z}{\mathbb{Z}}
\newcommand{\Q}{\mathbb{Q}}

\setlength{\parindent}{36pt}
\setlength{\parskip}{0em}

% \begin{align} A=B & Axiom 1    Puts equation on left and axiom on right

\begin{document}

\newtheorem*{prop10.27}{Proposition 10.27}
\begin{prop10.27}
	Given any $r \in \R_{>0}$, the number $\sqrt{r}$ is unique in the sense that, if x is a positive real number such that $x^2 = r$, then $x = \sqrt{r}$.
\end{prop10.27}

\begin{proof}
	Suppose u and v are such that $u^2 = r$ and $v^2 = r$. Let $w = sup\{x \in \R | x^2 < r\}$. We will show u = w = v. For any $x \in A := \{x \in \R | x^2 < r\}$ we see that $x^2 < u^2 = r$. If $x < 0$, then clearly $x < u$. If $x \geq 0$, then proposition 10.5 ($x < u$ if and only if $x^2 < u^2$) ensures that $x < u$. Since w is the least upper bound of A, we conclude that $w \leq u$. But $w^2 = r = u^2$. By proposition 10.5 again it must be the case that w = u. Similarly, v = w.
\end{proof}
\end{document}
