\documentclass[12pt]{amsart}

\usepackage{amsmath}
\usepackage{amssymb}
\usepackage{amsthm}
\usepackage{mathrsfs}
\usepackage{enumerate}

%All of these let you just type $\N$ for the $\N$atural numbers symbol and so on
\newcommand{\N}{\mathbb{N}}
\newcommand{\C}{\mathbb{C}}
\newcommand{\R}{\mathbb{R}}
\newcommand{\Z}{\mathbb{Z}}
\newcommand{\Q}{\mathbb{Q}}

\setlength{\parindent}{36pt}
\setlength{\parskip}{0em}

% \begin{align} A=B & Axiom 1    Puts equation on left and axiom on right

\begin{document}

\title{Homework 10}
\date{April 03, 2017}
\author{Leandro Ribeiro}

\maketitle

\newtheorem*{prop8.40}{Proposition 8.40}

\begin{prop8.40}
	(i) $x \in \R_{>0}$ if and only if $\frac{1}{x} \in \R_{>0}$.
	\center(ii) Let $x,y \in \R_{>0}$ . If $x < y$ then $0 < \frac{1}{y} < \frac{1}{x}$ .
\end{prop8.40}
\begin{proof}
	
	(i) ($\Rightarrow$) Let's first assume $x \in \R_{>0}$. We must show $\frac{1}{x} \in \R_{>0}$. By the axiom 8.26(iv) either $\frac{1}{x} = 0$, $\frac{1}{x} \in \R_{>0}$, or $-\frac{1}{x} \in \R_{>0}$.
	\\\textbf{Case 1.} $\frac{1}{x} = 0$. We know $1 \neq 0$ and we know $x \neq 0$ since $x \in \R_{>0}$. Let's consider $x \cdot \frac{1}{x}$. $1 = x \cdot \frac{1}{x} = x \cdot 0 = 0$, which is absurd.
	\\\textbf{Case 2.} $\frac{1}{x} \in \R_{>0}$. This is what we want, if this case holds, we're done.
	\\\textbf{Case 3.} Say $-\frac{1}{x} \in \R_{>0}$. Thus, $(-1)(\frac{1}{x}) \in \R_{>0}$. Since $x \in \R_{>0}$ and $(-1)(\frac{1}{x}) \in \R_{>0}$, $x \cdot (-1) \cdot (\frac{1}{x}) \in \R_{>0}$ since $\R_{>0}$ is closed under multiplication. Commuting, we see that $-1 \cdot 1 = -1 \in \R_{>0}$. On the other hand, $1 \in \R_{>0}$, so $1 + -1 = 0$ is in $\R_{>0}$, which is absurd.
	\\\indent We may now infer that case 2 must hold. Hence $\frac{1}{x} \in \R_{>0}$.\\
	\indent($\Leftarrow$)Now, let's assume $\frac{1}{x} \in \R_{>0}$. We must show $x \in \R_{>0}$. By the axiom 8.26(iv) either $x \in \R_{>0}$, $x = 0$, or $-x \in \R_{>0}$.
	\\\textbf{Case 1.} $x = 0$. We know $1 \neq 0$ and we know $\frac{1}{x} \neq 0$ since $\frac{1}{x} \in \R_{>0}$. Let's consider $\frac{1}{x} \cdot x$. $1 = \frac{1}{x} \cdot x = \frac{1}{x} \cdot 0 = 0$, which is absurd.
	\\\textbf{Case 2.} $x \in \R_{>0}$. This is what we want, if this case holds, we're done.
	\\\textbf{Case 3.} Say $-x \in \R_{>0}$. Thus, $(-1)(x) \in \R_{>0}$. Since $\frac{1}{x} \in \R_{>0}$ and $(-1)(x) \in \R_{>0}$, $\frac{1}{x} \cdot (-1) \cdot (x) \in \R_{>0}$ since $\R_{>0}$ is closed under multiplication. Commuting, we see that $-1 \cdot 1 = -1 \in \R_{>0}$. On the other hand, $1 \in \R_{>0}$, so $1 + -1 = 0$ is in $\R_{>0}$, which is absurd.
	\\\indent(ii) Consider $x < y$. Let's multiply each side by $\frac{1}{x}$. We obtain $1 < \frac{y}{x}$. By 8.40(i), $\frac{1}{x} \in \R_{>0}$; this is why $1 < \frac{y}{x}$. We can multiply both sides by $\frac{1}{y}$, and since $\frac{1}{y} \in \R_{>0}$, by 8.40(i), we have $\frac{1}{y} < \frac{1}{x}$. Finally, since $\frac{1}{y} \in \R_{>0}$, $0 < \frac{1}{y} < \frac{1}{x}$.
\end{proof}

\newtheorem*{thm8.41}{Theorem 8.41}
\begin{thm8.41}
	Let $x \in \R_{>0}$. Then $x^2 < x^3$ if and only if $x > 1$.
\end{thm8.41}

\begin{proof}
	($\Rightarrow$) We assume $x^2 < x^3$. Let's observe that $x^3 = x \cdot x^2$. On the other hand, $x^2 = x^2 \cdot 1$, so we have $1 \cdot x^2 < x \cdot x^2$. If $x^2 \in \R$, then $\frac{1}{x^2} \in \R_{>0}$. Since $x^2 < x^3$, we conclude that $x \neq 0$. Since $x \neq 0$, $x^2 \in \R_{>0}$. In this case, $\frac{1 \cdot x^2}{x^3} < \frac{x^3}{x^2}$, hence $1 < x$.
	\\($\Leftarrow$) Suppose $1 < x$. Since $x > 0$, we have $x \in \R_{>0}$. Thus $x^2 \in \R_{>0}$. Multiplying both sides by $x^2$, we deduce that $x^2 < x^3$.
\end{proof}

\newtheorem*{prop8.43}{Proposition 8.43}
\begin{prop8.43}
	Let $x,y \in \R$ such that $x < y$. There exists $z \in \R$ such that $x < z < y$.
\end{prop8.43}

\begin{proof}
	By definition of $<$, $0 < y - x$ and $y - x \in \R_{>0}$. Suppose toward a contradiction there's no $z \in \R$ such that $x < z < y$. Thus, there is no real number s such that $0 < s < y - x$. For every $w \in \R_{>0}$, it is then the case that $y - x \leq w$. Hence $y - x$ is the last element of $\R_{>0}$, which contradicts theorem 8.42.
\end{proof}

\newtheorem*{prop8.50}{Proposition 8.50}
\begin{prop8.50}
	If the sets A and B are bounded above and $A \subseteq B$, then $sup(A) \leq sup(B)$.
\end{prop8.50}

\begin{proof}
	Sup(B) is an upper bound for B. Since $A \subseteq B$, sup(B) is also an upper bound for A. Since sup(A) is the least upper bound, $sup(A) \leq sup(B)$.
\end{proof}

\newtheorem*{prop8.53}{Proposition 8.53}
\begin{prop8.53}
	Every nonempty subset of $\R$ that is bounded below has a greatest lower bound.
\end{prop8.53}

\begin{proof}
	Let $B \subseteq \R$ be a non-empty set. bounded from below. Consider A := \{$r \in \R$ $|$ $\forall b \in B$ $r \leq b\}$. A is the set of lower bounds of B. The set B is non-empty, so there is some $b \in B$. This b is clearly an upper bound for A. Hence, A admits a least upper bound. Applying our axiom, sup(A) exists. Consider $b \in B$. Either $sup(A) \leq b$ or $b < sup(A)$. Suppose toward a contradiction $b < sup(A)$. In this case, b is an upper bound of A. This contradicts that sup(A) is the least upper bound. We thus can eliminate $b < sup(A)$. Thus, for all $b \in B$, $sup(A) \leq b$. We conclude that sup(A) is a lower bound for B.
\end{proof}

\newtheorem*{lemma}{Lemma}
\begin{lemma}
	Let p be a prime. If $p | $($a_1 \dots a_{n}$), then $p | a_{i}$ for some $1 \leq i \leq n$.
\end{lemma}

\begin{proof}
	Suppose toward a contradiction that the lemma fails. Let $n \geq 1$ be least $\sum := $ \{ $n \in \N$ $|$ there are $a_1\dots a_n$ integers such that $p | a_1\dots a_n$ but $p \nmid a_1$ and $p \nmid a_2$ and \dots $p \nmid a_n$\}. By the well-ordering principle there is a least element (the set is non-empty since we assume the lemma fails).
	\\\indent By Euclid's lemma, we know that $n > 2$ (i.e. Euclid's lemma tells us the lemma holds for n = 2). Consider $p | a_1 \dots a_n$. We may rewrite $a_1\dots a_n = b \cdot c$ where b:= $a_1 \dots a_{n-1}$, and c := $a_n$. By Euclid's lemma, $p | b \cdot c$ implies $p | b$ or $p | c$. We know $p \nmid a_n$, so $p \nmid c$. Hence, $p | b$. But now $p | a_1 \dots a_{n-1}$ and $p \nmid a_1 \dots p \nmid a_{n-1}$. Since $n -1 \in \sum \in \N$, this contradicts that n is the least counter example. Thus, the lemma holds.
\end{proof}

\end{document}
