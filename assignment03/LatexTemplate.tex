\documentclass[12pt]{amsart}

\usepackage{amsmath}
\usepackage{amssymb}
\usepackage{amsthm}
\usepackage{mathrsfs}
\usepackage{enumerate}

%All of these let you just type $\N$ for the $\N$atural numbers symbol and so on
\newcommand{\N}{\mathbb{N}}
\newcommand{\C}{\mathbb{C}}
\newcommand{\R}{\mathbb{R}}
\newcommand{\Z}{\mathbb{Z}}
\newcommand{\Q}{\mathbb{Q}}

\setlength{\parindent}{36pt}
\setlength{\parskip}{0em}

% \begin{align} A=B & Axiom 1    Puts equation on left and axiom on right

\begin{document}

\title{Homework 3}
\date{February 6, 2017}
\author{Leandro Ribeiro\\(Worked with Kyle Franke and Joyce Gomez)}

\maketitle

\theoremstyle{plain}
\newtheorem*{axiom1.2}{Axiom 1.2}
\begin{axiom1.2}
There exists an integer 0 such that whenever m $\in$ $\Z$, m + 0 = m.
\end{axiom1.2}

\noindent
\textbf{Negation.} For all integers n there exists some m $\in$ $\Z$ such that m + n $\neq$ m

\newtheorem*{axiom1.3}{Axiom 1.3}
\begin{axiom1.3}
There exists an integer 1 such that 1 $\neq$ 0 and whenever m $\in$ $\Z$, m $\cdot$ 1 = m.
\end{axiom1.3}

\noindent
\textbf{Negation.} For all integers n, n = 0 and there exists some m $\in$ $\Z$ such that m $\cdot$ n $\neq$ m

\newtheorem*{axiom1.4}{Axiom 1.4}
\begin{axiom1.4}
For each m $\in$ $\Z$, there exists an integer, denoted by -m, such that m + (-m) = 0.
\end{axiom1.4}

\noindent
\textbf{Negation.} There exists some m $\in$ $\Z$ such that for all integers denoted by -m, m + (-m) $\neq$ 0.

\newtheorem*{axiom1.5}{Axiom 1.5}
\begin{axiom1.5}
Let m, n, and p be integers. If m $\cdot$ n = m $\cdot$ p and m $\neq$ 0, then n = p.
\end{axiom1.5}

\noindent
\textbf{Negation.} There exists some m, n, and p $\in$ $\Z$ such that m $\cdot$ n = m $\cdot$ p and m $\neq$ 0, and n $\neq$ p.

\newtheorem*{prop2.2}{Proposition 2.2}
\begin{prop2.2}
For m $\in$ $\Z$, one and only one of the following is true: 
\\(i) m $\in$ $\N$
\\(ii) -m $\in$ $\N$
\\(iii) m = 0.
\end{prop2.2}

\begin{proof}
Suppose first m = 0. By proposition 1.22, -0 = 0. Since 0 $\notin$ $\N$, we can conclude that -0 $\notin$ $\N$. Therefore both (ii) and (iii) fail. We may now assume that m $\neq$ 0. We argue by contradiction that m and -m are not both elements of $\N$. Suppose toward a  contradiction that m $\in$ $\N$ and -m $\in$ $\N$. By our axiom for $\N$, $\N$ is closed under addition. m + (-m) $\in$ $\N$. We infer that 0 $\in$ $\N$. By our axiom, 0 $\notin$ $\N$, so we have a contradiction. We conclude that m and -m are not both in $\N$.
\end{proof}

\newtheorem*{prop2.3}{Proposition 2.3}
\begin{prop2.3}
1 $\in$ $\N$
\end{prop2.3}

\begin{proof}
Suppose toward a contradiction 1 $\notin$ $\N$. By our axiom, either 1 $\in$ $\N$, 1 = 0, or -1 $\in$ $\N$. We know 1 $\neq$ 0 by our axioms for the integers. We conclude that either 1 $\in$ $\N$ or -1 $\in$ $\N$. Our reductio assumption implies -1 $\in$ $\N$. The set $\N$ is closed under multiplication. Thus (-1)n is a natural number for any n $\in$ $\N$. Then (-1)(-1) $\in$ $\N$. By proposition 1.20, (-1)(-1) = 1 $\cdot$ 1 = 1 by multiplicative identity. Therefore 1 $\in$ $\N$ by contradiction.
\end{proof}

\newtheorem*{prop2.7}{Proposition 2.7}
\begin{prop2.7}
Let m, n, p, q $\in$ $\Z$:
\\(i) If m $<$ n then m + p $<$ n + p.
\\(ii) If m $<$ n and p $<$ q then m + p $<$ n + q.
\\(iii) If 0 $<$ m $<$ n and 0 $<$ p $\leq$ q then mp $<$ nq.
\\(iv) If m $<$ n and p $<$ 0 then np $<$ mp.
\end{prop2.7}

\begin{proof}
(i) We observe n - m $\in$ $\N$ by definition. Consider \\n + p - (m + p) = n + p + -(m + p). By proposition 1.25.1, this is equal to n + p - m - p. We can rearrange this using commutativity to be n - m + p - p. After applying axiom 1.4, we know n - m + p - p = n - m + 0 = n - m by axiom 1.2. Since n - m $\in$ $\N$, we conclude that n + p - (m + p) $\in$ $\N$, so m + p $<$ n + p.
\\\indent(ii) Observe that n - m $\in$ $\N$ and q - p $\in$ $\N$. Consider \\n + q - (m + p) = n + q + -(m + p). By axiom 1.25.1, \\n + q + -(m + p) = n + q  - m - p. After applying commutativity and associativity, we can see that n + q  - m - p = (n - m) + (q - p). Since $\N$ is closed under addition and (n - m) $\in$ $\N$ and (q - p) $\in$ $\N$, we conclude (n - m) + (q - p) $\in$ $\N$. Therefore m + p $<$ n + q.
\\\indent(iii) We're given 0 $<$ m, m $<$ n, 0 $<$ p, and p $\leq$ q. Let's make several observations. Since 0 $<$ m, m - 0 $\in$ $\N$. As m - 0 = m by the identity element of addition, we conclude m $\in$ $\N$. Since m $<$ n, we also have that m - n $\in$ $\N$. Since 0 $<$ p, we have p - 0 $\in$ $\N$, so p $\in$ $\N$. We finally have that p $\leq$ q. Hence either p $<$ q or p = q. We here have two cases.
\\\textbf{Case 1:} p = q holds. Consider nq - mp. Observe that nq - mp = nq - mq. Factoring, we have that nq - mq = (n - m)q. By our observations, p $\in$ $\N$. Since $\N$ is closed under multiplication, (n - m)q $\in$ $\N$. We deduce that nq - mp $\in$ $\N$. Hence mp $<$ nq.
\\\textbf{Case 2:} p $<$ q holds. We have q - p $\in$ $\N$. Consider nq - mp. Clearly nq - mp = nq - mq + mq - mp. nq - mq + mq - mp = (n - m)q + m(q - p). We have that m $\in$ $\N$, n - m $\in$$\N$, and q - p $\in$ $\N$. We also see that 0 $<$ p $<$ q, so 0 $<$ q by transitivity of $<$, therefore q = q - 0 $\in$ $\N$. Since $\N$ is closed under multiplication and addition we infer that (n - m)q + m(q - p) $\in$ $\N$. Therefore, nq - mp $\in$ $\N$, so mp $<$ nq.
\\(iv) Let's make several observations.\ n - m $\in$ $\N$, and 0 - p $\in$ $\N$ by definition. Thus -p $\in$ $\N$. -p(n - m) $\in$ $\N$, because $\N$ is closed under multiplication. By distributivity, -pn + (-p)(-m) $\in$ $\N$. By proposition 1.25, -pn +(-p)(-m) = (-1)pn + (-1)p(-1)m $\in$ $\N$. By cor 1.21 (-1)(-1) = 1. We have that 1 $\cdot$ mp + -(np) $\in$ $\N$ by 1.25, we conclude that mp - np $\in$ $\N$. Therefore mp $<$ np.
\end{proof}

\newtheorem*{prop2.8}{Proposition 2.8}
\begin{prop2.8}
Let m,n $\in$ $\Z$. Exactly one of the following is true: m $<$ n, m = n, m $>$ n.
\end{prop2.8}

\begin{proof}
	Let's first observe the first case, m $<$ n. By definition, this means that n - m $\in$ $\N$. Proposition 2.2 states that -(n - m) $\notin$ $\N$ and n - m $\neq$ 0.
-(n - m) = -(n + (-m)) = (-1)(n + (-m)) by proposition 1.25. We can then distribute
and see that (-1)(n + (-m)) = (-1)n + (-1)(-m). After reapplying 1.25, (-1)n + (-1)(-m)
 = -n + m. -n + m = m - n by commutativity and the definition of subtraction. 
Therefore -(n - m) = (m - n) $\notin$ $\N$. This means m $\ngtr$ n. n - m $\neq$ 0 
tells us that m $\neq$ n, as they are not additive inverses.
	\\\indent In the event that m = n, we know that m - n = m - m = m + (-m) = 0 thanks to the additive inverse axiom. 2.2 also tells us that 
 m - n $\notin$ $\N$ and its negation, n - m $\notin$ $\N$. Therefore m $\ngtr$ n and m $\nless$ n.
	\\\indent Finally, in the event that m $>$ n we know that m - n $\in$ $\N$. Proposition 2.2 states that -(m - n) $\notin$ $\N$ and m - n $\neq$ 0.
-(m - n) = -(m + (-n)) = (-1)(m + (-n)) by proposition 1.25. We can then distribute
and see that (-1)(m + (-n)) = (-1)m + (-1)(-n). After reapplying 1.25, (-1)m + (-1)(-n)
 = -m + n. -m + n = n - m by commutativity and the definition of subtraction. 
Therefore -(m - n) = (n - m) $\notin$ $\N$. This means m $\nless$ n. m - n $\neq$ 0 
tells us that m $\neq$ n, as they are not additive inverses.
\end{proof}
\newtheorem*{prop2.10}{Proposition 2.10}
\begin{prop2.10}
The equation $x^{2}$ = -1 has no solution in $\Z$.
\end{prop2.10}

\begin{proof}
	Suppose towards a contradiction that $x^{2}$ = -1 has a solution in $\Z$. By definition, this means x $\cdot$ x = -1. Let m $\in$ $\Z$ and m $\neq$ 0. Suppose m $\in$ $\N$. Because $\N$ is closed under multiplication $m^{2}$ = m $\cdot$ m $\in$ $\N$. On the other hand, suppose -m $\in$ $\N$. By proposition 1.20, $m^{2}$ = m $\cdot$ m = (-m)(-m) $\in$ $\N$, thanks to $\N$ being closed under multiplication. We conclude $m^{2}$ $\in$ $\N$ for all m $\in$ $\Z$ if m $\neq$ 0. Therefore, x $\in$ $\Z$ which means -1 $\in$ $\N$, which is absurd since -1 $<$ 0, so it is a negative integer. By definition $\N$ only contains positive integers.
\end{proof}

\newtheorem*{prop2.12}{Proposition 2.12}
\begin{prop2.12}
For all m, n, p $\in$ $\Z$:
\\(i) -m $<$ -n if and only if m $>$ n.
\\(ii) If p $>$ 0 and mp $<$ np then m $<$ n.
\\(iii) If p $<$ 0 and mp $<$ np then n $<$ m.
\\(iv) If m $\leq$ n and 0 $\leq$ p then mp $\leq$ np. 
\end{prop2.12}
\begin{proof}
(i) Let's first prove that if -m $<$ -n, then m $>$ n. By definition, -n - (-m) $\in$ $\N$. That is, -n + -(-m) $\in$ $\N$. Appealing to prop. 1.22, -(-m) = m. We conclude 
that -n + m $\in$ $\N$. By commutativity, m - n $\in$ $\N$, hence n $<$ m. We need to prove the other implication. That is, if n $<$ m, then -m $<$ -n. By definition, m - n $\in$ $\N$. By commutativity, m - n = -n + m. Using proposition 1.22 m = -(-m), so -n + -(-m) $\in$ $\N$. By definition, -n - (-m) $\in$ $\N$. Therefore, -m $<$ -n.
\\\indent(ii) We have p $>$ 0, so p - 0 $\in$ $\N$ and therefore p $\in$ $\N$. By definition, np - mp $\in$ $\N$. np - mp = np + -mp. By proposition 1.2, np + -mp = (n + -m) $\cdot$ p. Appealing to commutativity, p(n - m) $\in$ $\N$. By proposition 2.11(proved below), we conclude that n - m $\in$ $\N$. Therefore m $<$ n.
\\\indent(iii)We have p $<$ 0, so  0 - p = 0 + (-p) $\in$ $\N$ and therefore (-p) $\in$ $\N$ by proposition 1.7. By definition, n(-p) - m(-p) $\in$ $\N$. n(-p) - m(-p) = n(-p) + -m(-p) = n(-p) + mp by proposition 1.20. By proposition 1.2 and 1.25, n(-p) + mp = -np + mp = (-n + m) $\cdot$ p. Appealing to commutativity, p(m  + -n) $\in$ $\N$. By proposition 2.11(proved below), we conclude that m - n $\in$ $\N$. Therefore n $<$ m.
\\(iv)Here we have several different cases to consider.
\\\textbf{Case 1.} p = 0. m $\cdot$ 0 = n $\cdot$ 0 = 0 by proposition 1.14, therefore mp = np.
\\\textbf{Case 2.} m = n. By applying the additive inverse, we can see that n - m = 0. We can multiply p on both sides on the left to get p(n - m) = p $\cdot$ 0 = 0. If we distribute we can see mp - np = 0. This means mp = np by definition.
\\\textbf{Case 3.} m $<$ n. By definition, this means n - m $\in$ $\N$. Because p - 0 = p $\in$ $\N$ by our axioms for the integers, p(n - m) $\in$ $\N$ because $\N$ is closed under multiplication. If we distribute and apply commutativity, p(n - m) = p(n + -m) = pn - pm =  np - mp $\in$ $\N$. This means mp $<$ np by definition.
\end{proof}
\newtheorem*{prop2.11}{Proposition 2.11}
\begin{prop2.11}
Let m $\in$ $\N$ and n $\in$ $\Z$. If m $\cdot$ n $\in$ $\N$, then n $\in$ $\N$.
\end{prop2.11}
\begin{proof}
	By proposition 2.2, exactly one of the following hold: n = 0, n $\in$ $\N$, or -n$\in$ $\N$. 
	\\\textbf{Case 1.} n = 0. Since n = 0, m $\cdot$ 0 $\in$ $\N$. By proposition 1.14, m $\cdot$ 0 = 0. Therefore, 0 $\in$ $\N$, which is absurd. We conclude that this is impossible.
	\\\textbf{Case 2.} n $\in$ $\N$. Since we're trying to prove this, we're done.
	\\\textbf{Case 3} -n $\in$ $\N$. Since m $\in$ $\N$, m $\cdot$ (-n) $\in$ $\N$ because $\N$ is closed under multiplication. m $\cdot$ n + m(-n) $\in$ $\N$ because $\N$ is closed under addition. Using distributivity, m(n + (-n)) $\in$ $\N$. Thus m $\cdot$ 0 $\in$ $\N$, but this implies 0 $\in$ $\N$, which is absurd.
\end{proof}
\end{document}
