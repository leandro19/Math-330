\documentclass[12pt]{amsart}

\usepackage{amsmath}
\usepackage{amssymb}
\usepackage{amsthm}
\usepackage{mathrsfs}
\usepackage{enumerate}

%All of these let you just type $\N$ for the $\N$atural numbers symbol and so on
\newcommand{\N}{\mathbb{N}}
\newcommand{\C}{\mathbb{C}}
\newcommand{\R}{\mathbb{R}}
\newcommand{\Z}{\mathbb{Z}}
\newcommand{\Q}{\mathbb{Q}}

\setlength{\parindent}{36pt}
\setlength{\parskip}{0em}

% \begin{align} A=B & Axiom 1    Puts equation on left and axiom on right

\begin{document}

\title{Challenge Problem 1}
\date{March 24, 2017}
\author{Leandro Ribeiro}

\maketitle

\theoremstyle{definition}
\newtheorem*{axD}{Axiom ($\dagger$)}
\begin{axD}
	There exists a subset $\N \subseteq \Z$ with the following properties:
	\\\indent(1) If $m,n \in \N$, then $m + n \in \N$.
	\\\indent(2) If $m,n \in \N$, then $mn \in \N$.
	\\\indent(3) $0 \notin \N$.
	\\\indent(4) For every $m \in \Z$, we have $m \in \N$, $m = 0$, or $- m \in \N$.
\end{axD}

\newtheorem*{definition}{Definition}
\begin{definition}
The \textbf{successor function} on the integers is defined by $s(x) :=
x + 1$. For $m \geq 1$, we define $s^{m+1}(x)$ recursively by $s^{m+1}(x) := s(s^{m}(x))$.
\end{definition}

\newtheorem*{ax1}{Axiom($\ast$)}
\begin{ax1}
For all $m \geq 1$, $s^{m}(0) \neq 0$.\vspace{2mm}
\end{ax1}

\newtheorem*{ax2}{Axiom($\ast\ast$)}
\begin{ax2}
	For all x $\in \Z$, there is $m \geq 0$ such that $s^m(x) = 0$ or $s^m(0) = x$.
\end{ax2}
\begin{definition}
	The set of \textbf{successors of zero} is defined to be \center$N := \{x \in \Z | \exists m \geq 1 s^m(0) = x\}$\vspace*{1.5mm}
\end{definition}
\noindent\hspace*{5mm} Prove the following proposition.\vspace*{1.5mm}

\theoremstyle{proposition}
\newtheorem*{proposition}{Proposition}
\begin{proposition}
	Assume that the integers satisfy \textnormal{($\ast$)} and \textnormal{($\ast\ast$)}. Then, the set
	N satisfies axiom \textnormal{($\dagger$)}. That is to say, N satisfies the four conditions of the
axiom.
\end{proposition}

\begin{proof}
	(1) Let P(n) be the statement "if $m \in N$, then $m + n \in N$".\\Because $m \in N$, there exists some $y \geq 1$ such that $ s^y(0) = m$. Let's first observe P(1)
	\\\textbf{Base.} n = 1. $m + 1 = s^y(0) + 1 = s(s^y(0))$, because $m \in N$ and the definition of the successor function. Also by definition of the successor function, $s(s^y(0)) = s^{y+1}(0)$. Clearly, $y+1 > y \geq 1$. Thus, for $m+1$ there exists $y+1 > 1$ such that $s^{y+1}(0) = m+1$, $m+1 \in N$, and the proposition holds.
	\\\textbf{Successor.} Assume P(n) holds. That is, $m + n \in N$. By definition, this means there exists some $y \geq 1$ such that $s^y(0) = m+n$. Consider $m + n + 1$. $m + n + 1 = s(m + n) = s(s^y(0))$ by definition of the successor function. By induction, we know $m + n = s^y(0) \in N$. Thus, $s(s^{y}(0)) = s^{y+1}(0) \in N$ since for $m + n + 1$ there exists $y + 1 > y \geq 1$ such that $s^{y+1}(0) = m + n + 1$. By the principal of mathematical induction, the proposition holds.
	\\\indent (2) Let P(n) be the statement " If $m \in N$, then $mn \in N$."Observe P(1):
	\\\textbf{Base.}n = 1. $m \cdot 1 = m \in N$.
	\\\textbf{Successor.} Assume P(n) holds. That is, $mn \in \N$. Consider $m(n+1)$. By our axioms for the integers, we may rewrite this as $mn + m$. By induction, $mn \in N$, and we already know $m \in N$. Thus, by the first part of this proposition and induction, $mn + n \in N$.
	\\\indent (3) By our axiom ($\ast$), we know $0 \notin N$.
	\\\indent (4) Take $x \in \Z$. By axiom ($\ast\ast$), there is $m \geq 0$ such that $s^m(x) = 0$ or $s^m(0) = x$. If $x$ is such that $s^m(0) = x$, then by our definition of $N$, $x \in N$. If $x = 0$, then we are done. If $s^m(x) = 0$, we must prove two lemmas to show $-x \in N$.
	\newtheorem*{lemma1}{Lemma}
	\begin{lemma1}
		For all $x,y \in \Z$, $s^m(x+y) = s^m(x) + y$.
	\end{lemma1}
	\begin{proof}
		Take P(m) to be the statement "$s^m(x+y) = s^m(x) + y$" Let's first observe m = 1
		\\\textbf{Base.} m = 1. $s^1(x+y) = (x + y) + 1 = (x + 1) + y = s^1(x) + y$.
		\\\textbf{Successor.} Assume P(m) holds. Consider $s^{m+1}(x+y)$. By definition of the successor function, we may rewrite this as $s(s^{m}(x+y))$. By induction, we have that $s(s^m(x) + y)$. By definition, it follows that $s(s^m(x) + y) = s^m(x) + y + 1 = (s^m(x) + 1) + y = s(s^m(x)) + y = s^{m+1}(x) + y$. Thus, by the Principal of Mathematical Induction the lemma holds. 
	\end{proof}
	\newtheorem*{lemma2}{Lemma}
	\begin{lemma2}
		For all $x \in \Z$ If $s^m(x) = 0$, then $-x \in N$.
	\end{lemma2}
	\begin{proof}
		Take P(m) to be the statement "If $s^m(x) = 0$, then $-x \in N$." Let's first observe m = 1.
		\\\textbf{Base.} $s^1(x) = x + 1 = 0$. If $x + 1 = 0$, we may cancel to find that $x = -1$, and $-(-1) = 1 = s(0) \in N$.
		\\\textbf{Successor.}  Assume P(m) holds. Observe $s^{m+1}(x) = 0$. $s^{m+1}(x) = s(s^m(x)) = s^m(x) + 1$. Applying the previous lemma, it follows that $ s^m(x) + 1 = s^m(x+1) = 0$. By induction, $-(x+1) \in \N$. Thus, the lemma and finally the proposition hold by the Principal of Mathematical Induction.
	\end{proof}

\end{proof}

\end{document}
