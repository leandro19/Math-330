\documentclass[12pt]{amsart}

\usepackage{amsmath}
\usepackage{amssymb}
\usepackage{amsthm}
\usepackage{mathrsfs}
\usepackage{enumerate}

%All of these let you just type $\N$ for the $\N$atural numbers symbol and so on
\newcommand{\N}{\mathbb{N}}
\newcommand{\C}{\mathbb{C}}
\newcommand{\R}{\mathbb{R}}
\newcommand{\Z}{\mathbb{Z}}
\newcommand{\Q}{\mathbb{Q}}

\setlength{\parindent}{36pt}
\setlength{\parskip}{0em}

% \begin{align} A=B & Axiom 1    Puts equation on left and axiom on right

\begin{document}

\title{Homework 11}
\date{May 9, 2017}
\author{Leandro Ribeiro}

\maketitle

\newtheorem*{prop}{Proposition}

\begin{prop}
	$\displaystyle{\sum\limits_{i=0}^{\infty}\frac{1}{i}}$ diverges.
\end{prop}

\begin{proof}
	Suppose that the harmonic series converges to $S$. By definition, this means for all $\epsilon > 0$, there exists an $N \in \N$ such that for all $n \geq N$, $|S - \sum\limits_{i=0}^{n}\frac{1}{i}| < \epsilon$
	\begin{equation*}
	\begin{split}
	S &= 1 + \frac{1}{2} + \frac{1}{3} + \frac{1}{4} + \frac{1}{5} + \frac{1}{6} + \frac{1}{7} + \frac{1}{8} + \dots
	 \\ &= \Bigg(1 + \frac{1}{2}\Bigg) + \Bigg(\frac{1}{3} + \frac{1}{4}\Bigg) + \Bigg(\frac{1}{5} + \frac{1}{6}\Bigg) + \Bigg(\frac{1}{7} + \frac{1}{8}\Bigg) + \dots
	 \\ &> \Bigg(\frac{1}{2} + \frac{1}{2}\Bigg) + \Bigg(\frac{1}{4} + \frac{1}{4}\Bigg) + \Bigg(\frac{1}{6} + \frac{1}{6}\Bigg) + \Bigg(\frac{1}{8} + \frac{1}{8}\Bigg) + \dots
	 \\ &= 1 + \frac{1}{2} + \frac{1}{3} + \frac{1}{4} + \frac{1}{5} + \frac{1}{6} + \frac{1}{7} + \frac{1}{8} + \dots
	 \\ &= S
	\end{split}
	\end{equation*}
	\indent Thus, $S > S$, which is absurd.
\end{proof}

\noindent\textbf{Sources.}
\\http://scipp.ucsc.edu/~haber/archives/physics116A10/harmapa.pdf

\end{document}
