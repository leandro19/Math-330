\documentclass[12pt]{amsart}

\usepackage{amsmath}
\usepackage{amssymb}
\usepackage{amsthm}
\usepackage{mathrsfs}
\usepackage{enumerate}

%All of these let you just type $\N$ for the $\N$atural numbers symbol and so on
\newcommand{\N}{\mathbb{N}}
\newcommand{\C}{\mathbb{C}}
\newcommand{\R}{\mathbb{R}}
\newcommand{\Z}{\mathbb{Z}}
\newcommand{\Q}{\mathbb{Q}}

\setlength{\parindent}{36pt}
\setlength{\parskip}{0em}

% \begin{align} A=B & Axiom 1    Puts equation on left and axiom on right

\begin{document}

\title{Challenge Problem 2}
\date{May 9, 2017}
\author{Leandro Ribeiro}

\maketitle

\theoremstyle{definition}
\newtheorem*{prop}{Project 4.23 (Leibinz's formula)}
\begin{prop}
	Consider an operation denoted by $'$ that is applied to symbols such as $u, v, w$. Assume that the operation $'$ satisfies the following axioms:
	\\\begin{center} $(u+v)' = u' + v'$
	\\ $(uv)' = uv' + u'v$
	\\ $(cu)' = cu'$, where $c$ is a constant.
	\end{center}
	Define $w^{(k)}$ recursively by
	\\\\(i) $w^{(0)} := w.$
	\\(ii) Assuming $w^{(n)}$ defined (where $n \in \Z_{\geq 0}$), define $w^{(n+1)} := (w^{(n)})'$
	\\
	Prove: $$(uv)^{(k)} = \sum_{m=0}^{k} {{k}\choose{m}} u^{(m)}v^{(k-m)}.$$
\end{prop}

\begin{proof}
	Take P(k) to be the statement "$(uv)^{(k)} = \sum_{m=0}^{k} {{k}\choose{m}} u^{(m)}v^{(k-m)}.$" Let's observe k=0.
	\\\textbf{Base.} $(uv)^{(0)} = \sum_{m=0}^{0} {{0}\choose{m}} u^{(m)}v^{(0-m)}. = u^{(0)}v^{(0)} = uv$
	\\\textbf{Successor.} Assume P(k) holds. Observe $\sum\limits_{m=0}^{k+1}u^{(m)}v^{(k+1 - m)}$. By proposition 4.16(i) we may rewrite this as $${{k+1}\choose{0}}u^{(0)}v^{(k+1)} + \sum\limits_{m=1}^{k} {{k+1}\choose{m}} u^{(m)}v^{(k+1-m)} +{{k+1}\choose{k+1}} u^{(k+1)}v^{0}.$$ By part (ii) of our recursive definition and Corollary 4.20, it follows that this is equal to $$uv^{(k+1)} + \sum\limits_{m=1}^{k} \Bigg({{k}\choose{m-1}} +{{k}\choose{m}}\Bigg) u^{(m)}v^{(k+1-m)} +u^{(k+1)}v$$. By distributivity and Proposition 4.16(ii), $$ = uv^{(k+1)} + \sum\limits_{m=1}^{k} {{k}\choose{m-1}}u^{(m)}v^{(k+1-m)} + \sum\limits_{m=1}^{k}{{k}\choose{m}} u^{(m)}v^{(k+1-m)} +u^{(k+1)}v$$
	Then, we apply proposition 4.17 to the first sum to get  $$ uv^{(k+1)} + \sum\limits_{m=0}^{k-1} {{k}\choose{m}}u^{(m+1)}v^{(k+1-(m+1))} + \sum\limits_{m=1}^{k}{{k}\choose{m}} u^{(m)}v^{(k+1-m)} +u^{(k+1)}v$$
	We then combine $uv^{(k+1)}$ to the second sum and $u^{(k+1)}v$ to the first using proposition 4.16(i): $$\sum\limits_{m=0}^{k} {{k}\choose{m}}u^{(m+1)}v^{(k-m)} + \sum\limits_{m=0}^{k}{{k}\choose{m}} u^{(m)}v^{(k+1-m)}$$
	Applying 4.16(ii) again and part (ii) of our recursive definition, we have $$\sum\limits_{m=0}^{k} {{k}\choose{m}}(u^{(m)})'v^{(k-m)} + {{k}\choose{m}} u^{(m)}(v^{(k-m)})'.$$ Using our second and third axioms for $'$, we thus have $$\sum\limits_{m=0}^{k} {{k}\choose{m}}(u^{(m)}v^{(k-m)})'.$$ Finally, by our first axiom for $'$, this is equal to $(\sum\limits_{m=0}^{k} {{k}\choose{m}}u^{(m)}v^{(k-m)})'$, and by induction and our recursive definition $(\sum\limits_{m=0}^{k} {{k}\choose{m}}u^{(m)}v^{(k-m)})' = ((uv)^{(k)})' = (uv)^{(k+1)}$. This completes the induction.
\end{proof}

\end{document}
