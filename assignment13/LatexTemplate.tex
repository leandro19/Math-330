\documentclass[12pt]{amsart}

\usepackage{amsmath}
\usepackage{amssymb}
\usepackage{amsthm}
\usepackage{mathrsfs}
\usepackage{enumerate}

%All of these let you just type $\N$ for the $\N$atural numbers symbol and so on
\newcommand{\N}{\mathbb{N}}
\newcommand{\C}{\mathbb{C}}
\newcommand{\R}{\mathbb{R}}
\newcommand{\Z}{\mathbb{Z}}
\newcommand{\Q}{\mathbb{Q}}

\setlength{\parindent}{36pt}
\setlength{\parskip}{0em}

% \begin{align} A=B & Axiom 1    Puts equation on left and axiom on right

\begin{document}

\title{Homework 12}
\date{April 24, 2017}
\author{Leandro Ribeiro}

\maketitle

\newtheorem*{prop10.27}{Proposition 10.27}
\begin{prop10.27}
	Given any $r \in \R_{>0}$, the number $\sqrt{r}$ is unique in the sense that, if x is a positive real number such that $x^2 = r$, then $x = \sqrt{r}$.
\end{prop10.27}

\begin{proof}
	Suppose u and v are such that $u^2 = r$ and $v^2 = r$. Let $w = sup\{x \in \R | x^2 < r\}$. We will show u = w = v. For any $x \in A := \{x \in \R | x^2 < r\}$ we see that $x^2 < u^2 = r$. If $x < 0$, then clearly $x < u$. If $x \geq 0$, then propositio 10.5 ensures that $x < u$. Since w is the least upper bound of A, we conclude that $w \leq u$. But $w^2 = r = u^2$. By proposition 10.5 again it must be the case that w = u. Similarly, v = w.
\end{proof}

\newtheorem*{prop11.12}{Proposition 11.12}

\begin{prop11.12}
	If $r \in \N$ is not a perfect square, then $\sqrt{r}$ is irrational.
\end{prop11.12}
Let us prove the contrapositive. Suppose $\sqrt{r}$ is rational. We aregue $r$ is a perfect square. Suppose $\sqrt{r} = \frac{m}{n}$ with $m$ and $n$ in lowest terms, i.e. the $gcd(m,n)=1$. Thus, $r=\frac{m^2}{n^2}$ and $rn^2 = m^2$. If p is prime and $k \geq 0$ is such that $p^k|n$ then $p^{2k}|n^2$. Since $rn^2 = m^2$, we conclude that $p^{2k}|m^2$. Thus, $p^k|m$. Since $n = p_1^{k_1}\dots p_l^{k_l}$, take $m = q_1^{a_1}\dots q_j^{a_j}$ with $q_i=p_i$ and $k_i \leq a_i$. $m = p_1^{k_1} \dots p_l^{k_l} \cdot c$ with $c = q_{l+1}^{a_{l+1}} \dots q_{j}^{a_j}$. Thus, $m = n \cdot c$. We conclude $n | m$, however $gcd(m,n)=1$, thus $n=1$. Hence, $\sqrt{r} = m$, and $r$ is a perfect square.
\begin{proof}

\end{proof}

\newtheorem*{prop11.4}{Proposition 11.4}
\begin{prop11.4}
	Given a rational number $r \in \Q$, we can always write it as $r = \frac{m}{n}$, where $n > 0$ and m and n do not have any common factors.
\end{prop11.4}

\begin{proof}
	Suppose toward a contradiction there were $m,n \in \Z_{>0}$ such that the fraction $\frac{m}{n}$ cannot be written in lowest terms. Let $C$ be the the set of positive integers that are numerators of such fractions. Then $m \in C$, so $C$ is not empty. Therefore, by the well-ordering principle there must be a smallest integer $m$ in $C$. There is an integer $n_0 > 0$ such that the fraction $\frac{m_0}{n_0}$ cannot be written in lowest terms by our definition of $C$. This means that $m_0$ and $n_0$ must have a common factor $p > 1$, however $(\frac{m_0}{p})/(\frac{n_0}{p}) = \frac{m_0}{n_0}$. Any way of expressing the left hand fraction in lowest terms would also work for $\frac{m_0}{n_0}$, which implies the fraction $(\frac{m_0}{p})/(\frac{n_0}{p})$ cannot be written in lowest terms either. By our definition of $C$, $\frac{m_0}{p}$, is in $C$, but $\frac{m_0}{p} < m_0$, which contradicts that $m_0$ is the smallest element of $C$.
\end{proof}

\newtheorem*{prop11.13}{proposition 11.13}
\begin{prop11.13}
	Let m and n be nonzero integers. Then $\frac{m}{n}\sqrt{2}$ is irrational.
\end{prop11.13}

\begin{proof}
	Suppose toward a contradiction $\frac{m}{n}\sqrt{2} \in \Q$. From our assumption, we may take $q,p \in \Z$ such that $\frac{m}{n}\sqrt{2} = \frac{p}{q}$. Multiplying by $\frac{n}{m}$ on both sides, we have $\frac{n}{m}\frac{m}{n}\sqrt{2} = \sqrt{2} = \frac{n}{m}\frac{p}{q}=\frac{np}{mq}$. Let $a = np $ and $b = mq$. Let's assume, by proposition 11.4, $\frac{a}{b}$ is in lowest terms. This means $gcd(a,b)=1$. We see that $\frac{(a)^2}{(b)^2} = 2$, so $a^2 = 2b^2$. We deduce that a is even. $2 | a \cdot a$, and by Euclid's lemma, $2 | a$. Since $2 | a$, we see $a = 2 \cdot k$. Hence $(2k)^2 = 2b^2 = 4k^2$. Therefore, $2k^2 = b^2$. As with $a$, we conclude that $2 | b$. This is absurd since we assumed $gcd(a,b) = 1$.
\end{proof}

\textbf{Sources.}
\\https://math.stackexchange.com/questions/463342/prove-that-theres-no-fractions-that-cant-be-written-in-lowest-term-with-well-o
\end{document}
