\documentclass[12pt]{amsart}

\usepackage{amsmath}
\usepackage{amssymb}
\usepackage{amsthm}
\usepackage{mathrsfs}
\usepackage{enumerate}

%All of these let you just type $\N$ for the $\N$atural numbers symbol and so on
\newcommand{\N}{\mathbb{N}}
\newcommand{\C}{\mathbb{C}}
\newcommand{\R}{\mathbb{R}}
\newcommand{\Z}{\mathbb{Z}}
\newcommand{\Q}{\mathbb{Q}}

\setlength{\parindent}{36pt}
\setlength{\parskip}{0em}

% \begin{align} A=B & Axiom 1    Puts equation on left and axiom on right

\begin{document}

\title{Homework 12}
\date{April 24, 2017}
\author{Leandro Ribeiro}

\maketitle

\newtheorem*{prop10.27}{Proposition 10.27}
\begin{prop10.27}
	Given any $r \in \R_{>0}$, the number $\sqrt{r}$ is unique in the sense that, if x is a positive real number such that $x^2 = r$, then $x = \sqrt{r}$.
\end{prop10.27}

\begin{proof}
\end{proof}

\newtheorem*{prop11.12}{Proposition 11.12}

\begin{prop11.12}
	If $r \in \N$ is not a perfect square, then $\sqrt{r}$ is irrational.
\end{prop11.12}
\begin{proof}

\end{proof}

\newtheorem*{prop11.4}{Proposition 11.4}
\begin{prop11.4}
	Given a rational number $r \in \Q$, we can always write it as $r = \frac{m}{n}$, where $n > 0$ and m and n do not have any common factors.
\end{prop11.4}

\begin{proof}
\end{proof}

\newtheorem*{prop11.13}{proposition 11.13}
\begin{prop11.13}
	Let m and n be nonzero integers. Then $\frac{m}{n}\sqrt{2}$ is irrational.
\end{prop11.13}

\begin{proof}
\end{proof}

\textbf{Sources.}
\\
\end{document}
