\documentclass[12pt]{amsart}

\usepackage{amsmath}
\usepackage{amssymb}
\usepackage{amsthm}
\usepackage{mathrsfs}
\usepackage{enumerate}

%All of these let you just type $\N$ for the $\N$atural numbers symbol and so on
\newcommand{\N}{\mathbb{N}}
\newcommand{\C}{\mathbb{C}}
\newcommand{\R}{\mathbb{R}}
\newcommand{\Z}{\mathbb{Z}}
\newcommand{\Q}{\mathbb{Q}}

\setlength{\parindent}{36pt}
\setlength{\parskip}{0em}

% \begin{align} A=B & Axiom 1    Puts equation on left and axiom on right

\begin{document}

\title{Homework 6}
\date{February 27, 2017}
\author{Leandro Ribeiro\\(Worked with Kyle Franke)}

\maketitle

\newtheorem*{prop4.18}{Proposition 4.18}
\begin{prop4.18}
	Let $(x_j)_{j=1}^{\infty} and (y_j)_{j=1}^{\infty}$ be sequences in $\Z$ such that $x_j \leq y_j$ for all j $\in \N$. Then for all k $\in \N$,
	$\sum\limits_{j=1}^{k} x_j \leq \sum\limits_{j=1}^{k} y_j$.
\end{prop4.18}

\begin{proof}
	Let P(k) be the statement, "$\sum\limits_{j=1}^{k} x_j \leq \sum\limits_{j=1}^{k} y_j$." Let's observe P(1).
	\\\textbf{Base.} $\sum\limits_{j=1}^{1} x_j \leq \sum\limits_{j=1}^{1} y_j$. $x_1 \leq y_1$, thus P(1) holds.
	\\\textbf{Successor.} Assume P(n) holds. That is, $\sum\limits_{j=1}^{n} x_j \leq \sum\limits_{j=1}^{n} y_j$. Consider $\sum\limits_{j=1}^{n+1} x_j$ and $\sum\limits_{j=1}^{n+1} y_j$. By definition, we can rewrite this as $\sum\limits_{j=1}^{n} x_j + x_{n+1}$ and $\sum\limits_{j=1}^{n} y_j + y_{n+1}$. By induction, we know $\sum\limits_{j=1}^{n} x_j \leq \sum\limits_{j=1}^{n} y_j$, and we know $x_{n+1} \leq y_{n+1}$, thus by proposition 2.7(ii), $\sum\limits_{j=1}^{n+1} x_j \leq \sum\limits_{j=1}^{n+1} y_j$. We have proven that the proposition holds by induction.
\end{proof}

\newtheorem*{prop4.30}{Proposition 4.30}
\begin{prop4.30}
	For all k,m $\in \N$, where m $\geq 2$,\\\indent $f_{m+k} = f_{m-1}f_{k} + f_mf_{k+1}$.
\end{prop4.30}

\begin{proof}
	Let P(k) be the statement "$f_{m+k} = f_{m-1}f_{k} + f_mf_{k+1}$." Let's observe P(1) and P(2).
	\\\textbf{Base 1.}k = 1. $f_{m-1}f_{1} + f_mf_{2}$. By definition, we can rewrite this as $f_{m-1} \cdot 1 + f_m \cdot 1 = f_{m-1} + f_{m} = f_{m+1}$. Thus P(1) holds.
	\\\textbf{Base 2.}k = 2. $f_{m-1}f_{2} + f_mf_{3}$. By definition, we can rewrite this as $f_{m-1} \cdot 1 + f_m \cdot 2 = f_{m-1} + f_{m} + f_{m} = f_{m+1} + f_{m} = f_{m+2}$. Thus P(2) holds.
	\\\textbf{Successor.} Assume P(k) holds for all k = 1,2,\dots,n for some n $\geq 2$. That is,$f_{m+k} = f_{m-1}f_{k} + f_mf_{k+1}$. Consider \\$f_{m-1}f_{(n+1)} + f_mf_{(n+1)+1} = f_{m-1}f_{(n+1)} + f_mf_{n+2}$. By definition, we can rewrite this as $f_{m-1}(f_{n} + f_{n-1}) + f_m(f_{n+1} + f_{n})$. After distributing and commuting, we have $f_{m-1}(f_{n} + f_{n-1}) + f_m(f_{n+1} + f_{n}) = f_{m-1}f_{n} + f_{m-1}f_{n-1} + f_mf_{n+1} + f_mf_{n} = (f_{m-1}f_{n} + f_mf_{n+1}) + (f_{m-1}f_{n-1} + f_mf_{n}) = f_{m+n} + f_{m+n-1}$ by induction. By definition we have $f_{m+n} + f_{m+n-1} = f_{m+(n+1)}$. Thus P(n+1) holds and we have proven the proposition by the principle of induction.
\end{proof}

\newtheorem*{prop4.31}{Proposition 4.31}
\begin{prop4.31}
	For all k $\in \N$, $f_{2k+1} = f_k^2 + f_{k+1}^2$.
\end{prop4.31}

\begin{proof}
	Let P(k) be the statement "$f_{2k+1} = f_k^2 + f_{k+1}^2$." Let's observe P(1).
	\\\textbf{Base.} k=1. $f_{2+1} = f_{3} =f_1^2 + f_{1+1}^2 = 1 + 1 = 2$. Thus P(1) holds.
	\\\textbf{Successor.} Assume P(k) holds for k=1,2,\dots,n for some n $\in$ $\N$. Consider $f_{(n+1)}^2 + f_{(n+1)+1}^2 = f_{(n+1)}^2 + f_{(n+2)}^2$. By definition we can rewrite this as $(f_{n+1} + f_{n})^2 + f_{(n+1)}^2 = f_{n+1}^{2} + 2f_{n+1}f_{n} + f_{n}^{2} + f_{n+1}^2$. By induction, we have $f_{n+1}^{2} + 2f_{n+1}f_{n} + f_{n}^{2} + f_{n+1}^2 = f_{n+1}^{2} + f_{n}^{2} + 2f_{n+1}f_{n} + f_{n+1}^2 = f_{2n+1} + 2f_{n+1}f_{n} + f_{n+1}^2$. If we continue the computation, $f_{2n+1} + 2f_{n+1}f_{n} + f_{n+1}^2 = f_{2n+1} + f_{n+1}(2f_{n} + f_{n+1}) = f_{2n+1} + f_{n+1}(f_{n} + f_{n} + f_{n+1})$. By definition, $f_{2n+1} + f_{n+1}(f_{n} + f_{n} + f_{n+1}) = f_{2n+1} + f_{n+1}(f_{n} + f_{n+2}) = f_{2n+1} + f_{n+1}f_{n} + f_{n+1}f_{n+2}$. By proposition 4.30, we have $f_{2n+1} + f_{n+1}f_{n} + f_{n+1}f_{n+2} = f_{2n+1} + f_{2n+2} = f_{2n+3}$ by definition. Thus, P(n+1) holds and we have proven the proposition by induction.
\end{proof}

\newtheorem*{prop5.1}{Proposition 5.1}
\begin{prop5.1}
	Let A,B,C be sets.
	\\(i) $A \subseteq A.$
	\\(ii) If $A \subseteq B$ and $B \subseteq C$ then $A \subseteq C$.
\end{prop5.1}
(i) For any x $\in A$, then $x \in A$. Therefore A $\subseteq$ A.
\\\indent(ii) Let $x \in A$, since $A \subseteq B$ we get x $\in$ B. Since $B \subseteq C$, this implies $x \in C$.
\begin{proof}
\end{proof}

\newtheorem*{prop5.4}{Proposition 5.4}
\begin{prop5.4}
	Let A,B,C be sets.
	\\(i) A = A.
	\\(ii) if A = B then B = A.
	\\(iii) if A = B and B = C then A = C.
\end{prop5.4}

\begin{proof}
	(i) For any x $\in A$, then $x \in A$. Therefore A $\subseteq$ A. Also, for any x $\in A$, then $x \in A$. Therefore A $\subseteq$ A. Thus, A = A.
\\\indent(ii) It is given to us that $A \subseteq B$ and $B \subseteq A$. For any x $\in B$, then $x \in A$. Therefore B $\subseteq$ A. Also, for any x $\in A$, then $x \in B$. Therefore A $\subseteq$ B. Thus, B = A.
\\\indent(iii) It is given to us that $A \subseteq B$ and $B \subseteq A$. It is also given to us that $B \subseteq C$ and $C \subseteq B$. Because $A \subseteq B$ and $B \subseteq C$, we have that $A \subseteq C$ by proposition 5.1. Because $C \subseteq B$ and $B \subseteq A$, we have that $C \subseteq A$ by proposition 5.1. Thus by definition we have that A = C.
\end{proof}
\noindent\textbf{Sources.}
\\http://zimmer.csufresno.edu/~sdelcroix/sol111home6.pdf
\end{document}
