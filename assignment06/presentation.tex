\documentclass[12pt]{amsart}

\usepackage{amsmath}
\usepackage{amssymb}
\usepackage{amsthm}
\usepackage{mathrsfs}
\usepackage{enumerate}

%All of these let you just type $\N$ for the $\N$atural numbers symbol and so on
\newcommand{\N}{\mathbb{N}}
\newcommand{\C}{\mathbb{C}}
\newcommand{\R}{\mathbb{R}}
\newcommand{\Z}{\mathbb{Z}}
\newcommand{\Q}{\mathbb{Q}}

\setlength{\parindent}{36pt}
\setlength{\parskip}{0em}

% \begin{align} A=B & Axiom 1    Puts equation on left and axiom on right

\begin{document}
\newtheorem*{prop4.31}{Proposition 4.31}
\begin{prop4.31}
	For all k $\in \N$, $f_{2k+1} = f_k^2 + f_{k+1}^2$.
\end{prop4.31}

\begin{proof}
	Let P(k) be the statement "$f_{2k+1} = f_k^2 + f_{k+1}^2$." Let's observe P(1).
	\\\textbf{Base.} k=1. $f_{2+1} = f_{3} =f_1^2 + f_{1+1}^2 = 1 + 1 = 2$. Thus P(1) holds.
	\\\textbf{Successor.} Assume P(k) holds for k=1,2,\dots,n for some n $\in$ $\N$. Consider $f_{(n+1)}^2 + f_{(n+1)+1}^2 = f_{(n+1)}^2 + f_{(n+2)}^2$. By definition we can rewrite this as $(f_{n+1} + f_{n})^2 + f_{(n+1)}^2 = f_{n+1}^{2} + 2f_{n+1}f_{n} + f_{n}^{2} + f_{n+1}^2$. By induction, we have $f_{n+1}^{2} + 2f_{n+1}f_{n} + f_{n}^{2} + f_{n+1}^2 = f_{n+1}^{2} + f_{n}^{2} + 2f_{n+1}f_{n} + f_{n+1}^2 = f_{2n+1} + 2f_{n+1}f_{n} + f_{n+1}^2$. If we continue the computation, $f_{2n+1} + 2f_{n+1}f_{n} + f_{n+1}^2 = f_{2n+1} + f_{n+1}(2f_{n} + f_{n+1}) = f_{2n+1} + f_{n+1}(f_{n} + f_{n} + f_{n+1})$. By definition, $f_{2n+1} + f_{n+1}(f_{n} + f_{n} + f_{n+1}) = f_{2n+1} + f_{n+1}(f_{n} + f_{n+2}) = f_{2n+1} + f_{n+1}f_{n} + f_{n+1}f_{n+2}$. By proposition 4.30, we have $f_{2n+1} + f_{n+1}f_{n} + f_{n+1}f_{n+2} = f_{2n+1} + f_{2n+2} = f_{2n+3}$ by definition. Thus, P(n+1) holds and we have proven the proposition by induction.
\end{proof}
\end{document}
