\documentclass[12pt]{amsart}

\usepackage{amsmath}
\usepackage{amssymb}
\usepackage{amsthm}
\usepackage{mathrsfs}
\usepackage{enumerate}

%All of these let you just type $\N$ for the $\N$atural numbers symbol and so on
\newcommand{\N}{\mathbb{N}}
\newcommand{\C}{\mathbb{C}}
\newcommand{\R}{\mathbb{R}}
\newcommand{\Z}{\mathbb{Z}}
\newcommand{\Q}{\mathbb{Q}}

\setlength{\parindent}{36pt}
\setlength{\parskip}{0em}

% \begin{align} A=B & Axiom 1    Puts equation on left and axiom on right

\begin{document}

\title{Homework 11}
\date{April 18, 2017}
\author{Leandro Ribeiro}

\maketitle

\newtheorem*{prop9.12}{Proposition 9.12}

\begin{prop9.12}
	Let A and B be sets. There exists an injection from A to B if and only if there exists a surjection from B to A.
\end{prop9.12}
\begin{proof}
	($\Rightarrow$) Suppose there's an injection $f: A \rightarrow B$. Fix $a_{o} \in A$. Observe that for $b \in f(a)$ there's a unique $a_b \in A$ such that $f(a_b) = b$. Define $g: B \rightarrow A$ by $g(b) =\begin{cases}
		a_o \textrm{,  }b \notin f(a)\\
		a_b \textrm{,  }b \in f(a) 
	\end{cases}$is surjective by definition.
	\\\indent($\Leftarrow$) Suppose $g: B \rightarrow A$ is surjective. For each $a \in A$, there is at least one $b \in B$ such that $g(b) =  a$. For each $a \in A$, fix some such $b_a \in B$. Define $f: A \rightarrow B$ by $f(a) = b_a$. Let's check if f is injective. Suppose $a_1 \neq a_2$. Then, $g(a_1) \neq g(a_2)$.
\end{proof}

\newtheorem*{prop10.9}{Proposition 10.9}
\begin{prop10.9}
	Let $x \in \R$ be such that $0 \leq x \leq 1$, and let $m, n \in \N$ be such that $m \geq n$. Then $x^{m} \leq x^{n}$.
\end{prop10.9}

\begin{proof}
	Here we have three cases. $x = 0$, $x = 1$, and $0 < x < 1$.
	\\\textbf{Case 1.} x = 0. $0^m = 0 = 0^n$.
	\\\textbf{Case 2.} x = 1. $1^m = 1 = 1^n$
	\\\textbf{Case 3.} $0 < x < 1$. We have two different subcases here, since either $m = n$ or $m > n$.
	\\\indent\textbf{Subcase 1.}  m = n. Here $x^m = x^n$
	\\\indent\textbf{Subcase 2.} $m > n$. Take $y > 1$ such that $\frac{1}{y} = x$. Because $m > n$, we have $y^m > y^n$. By proposition 8.40 (ii), we know $\frac{1}{y^m} < \frac{1}{y^n}$. Thus, $\frac{1}{y^m} = \frac{1^m}{y^m} = (\frac{1}{y})^m = x^m < \frac{1}{y^n} = \frac{1^n}{y^n} = (\frac{1}{y})^n = x^n$.
\end{proof}

\newtheorem*{prop10.16}{Proposition 10.16}
\begin{prop10.16}
	If the sequence $(x_k)$ converges to L, then $$\lim_{k \to \infty} x_{k+1} = L.$$
\end{prop10.16}

\begin{proof}
\end{proof}

\newtheorem*{prop10.14}{Proposition 10.14}
\begin{prop10.14}
	If $(x_k)$ converges to L and to $L'$ then $L = L'$.
\end{prop10.14}

\begin{proof}
\end{proof}
\end{document}
