\documentclass[12pt]{amsart}

\usepackage{amsmath}
\usepackage{amssymb}
\usepackage{amsthm}
\usepackage{mathrsfs}
\usepackage{enumerate}

%All of these let you just type $\N$ for the Natural numbers symbol and so on
\newcommand{\N}{\mathbb{N}}
\newcommand{\C}{\mathbb{C}}
\newcommand{\R}{\mathbb{R}}
\newcommand{\Z}{\mathbb{Z}}
\newcommand{\Q}{\mathbb{Q}}

\setlength{\parindent}{36pt}
\setlength{\parskip}{0em}

% \begin{align} A=B & Axiom 1    Puts equation on left and axiom on right

\begin{document}

\title{Homework 1}
\date{January 23, 2017}
\author{Leandro Ribeiro\\(Worked with Kyle Franke and Joyce Gomez)}

\maketitle

\theoremstyle{plain}
\newtheorem*{prop1.7}{Proposition 1.7}
\begin{prop1.7}
	If m is an integer, then 0 + m = m and 1 $\cdot$ m = m.
\end{prop1.7}

\begin{proof}
	By commutativity of addition (axiom 1.1.1), 0 + m = m + 0. Because of the identity element for addition (axiom 1.2), we have that m + 0 = m. We conclude that 0 + m = m + 0.
	\\ 
	\indent By commutativity of multiplication (axiom 1.1.4), 1 $\cdot$ m = m $\cdot$ 1. Because of the identity element for multiplication (axiom 1.3), we have that m $\cdot$ 1 = m. We conclude that 1 $\cdot$ m = m $\cdot$ 1
	\\
\end{proof}
\newtheorem*{prop1.8}{Proposition 1.8}
\begin{prop1.8}
	If m is an integer, then (-m) + m = 0
\end{prop1.8}

\begin{proof}
	By commutativity of addition (axiom 1.1.1), m + (-m) = (-m) + m. Since m + (-m) = 0 by axiom 1.4 (additive inverse), we can conclude (-m) + m = 0.
	\\
\end{proof}

\newtheorem*{prop1.10}{Proposition 1.10}
\begin{prop1.10}
	Let m, $x_{1}$ , $x_{2}$ $\in$ \textbf{Z}. If m, $x_{1}$ , $x_{2}$ satisfy the equations m + $x_{1}$ = 0 and	m + $x_{2}$ = 0, then $x_{1}$ = $x_{2}$ .
\end{prop1.10}

\begin{proof}
	Because both equations are equal to 0, we know that m + $x_{1}$ = 0 = m + $x_{2}$. By proposition 1.9, we can say that $x_{1}$ = $x_{2}$.
	\\
\end{proof}

\newtheorem*{prop1.12}{Proposition 1.12}
\begin{prop1.12}
	Let x $\in$ \textbf{Z}. If x has the property that for each integer m, m + x = m,
then x = 0.
\end{prop1.12}

\begin{proof}
	Let's add (-m) to both sides from the left: (-m) + (m + x) = (-m) + m. By associativity ((-m) + m) + x = (-m) + m. By proposition 1.8, (-m) + m = 0, so 0 + x = 0. Proposition 1.7 tells us that 0 + x = x, so we vonclude x = 0.
	\\
\end{proof}

\newtheorem*{prop1.13}{Proposition 1.13}
\begin{prop1.13}
	Let x $\in$ \textbf{Z}. If x has the property that there exists an integer m such that m + x = m,
then x = 0.
\end{prop1.13}

\begin{proof}
	The proof for proposition 1.12 also proves this proposition.
	\\
\end{proof}

\newtheorem*{prop1.16}{Proposition 1.16}
\begin{prop1.16}
	If m and n are even integers, then so are m + n and mn.
\end{prop1.16}

\begin{proof}
	Since m and n are even there is $j_{1}$ and $j_{2}$ such that $j_{1}$ $\cdot$ 2 = m and $j_{2}$ $\cdot$ 2 = n. We now see that m + n = $j_{1}$ $\cdot$ 2 + $j_{2}$ $\cdot$ 2 = ($j_{1}$ + $j_{2}$) $\cdot$ 2 thanks to proposition 1.6. Also, m $\cdot$ n = ($j_{1}$ $\cdot$ 2)$\cdot$($j_{2}$ $\cdot$ 2) = ($j_{1}$ $\cdot$ 2 $\cdot$ $j_{2}$) $\cdot$ 2 thanks to associativity of multiplication (axiom 1.1.5). Setting integers d := $j_{1}$ $\cdot$ 2 $\cdot$ $j_{2}$ and c := $j_{1}$ + $j_{2}$, we have that d $\cdot$ 2 = m $\cdot$ n and c $\cdot$ 2 = m + n. We conclude that 2 $\mid$ m + n and 2 $\mid$ m $\cdot$ n.
	\\
\end{proof}

\newtheorem*{prop1.17}{Proposition 1.17}
\begin{prop1.17}
    (i) 0 is divisible by every integer.
	\\(ii) If m is an integer not equal to 0, then m is not divisible by 0.
\end{prop1.17}

\begin{proof}
	(i) By proposition 1.14, we have that 0 $\cdot$ m = 0 for any integer m. Since 0 is an integer, we conclude that m $\mid$ 0.
	\\
	\indent(ii) We assume m is divisible by 0. Then There is an integer j $\in$ \textbf{Z} such that j $\cdot$ 0 = m, but j $\cdot$ 0 = 0 by proposition 1.14, so m = 0.
	\\
\end{proof}

\newtheorem*{prop1.18}{Proposition 1.18}
\begin{prop1.18}
	Let x $\in$ \textbf{Z}. If x has the property that for all m $\in$ \textbf{Z}, mx = m, then x = 1.
\end{prop1.18}

\begin{proof}
	By axiom 1.3, we have that m $\cdot$ x = m = m $\cdot$ 1. By cancellation (axiom 1.5), we can conclude x = 1.
	\\
\end{proof}
\end{document}
