\documentclass[12pt]{amsart}

\usepackage{amsmath}
\usepackage{amssymb}
\usepackage{amsthm}
\usepackage{mathrsfs}
\usepackage{enumerate}

%All of these let you just type $\N$ for the $\N$atural numbers symbol and so on
\newcommand{\N}{\mathbb{N}}
\newcommand{\C}{\mathbb{C}}
\newcommand{\R}{\mathbb{R}}
\newcommand{\Z}{\mathbb{Z}}
\newcommand{\Q}{\mathbb{Q}}

\setlength{\parindent}{36pt}
\setlength{\parskip}{0em}

% \begin{align} A=B & Axiom 1    Puts equation on left and axiom on right

\begin{document}

\title{Homework 9}
\date{March 27, 2017}
\author{Leandro Ribeiro\\(Worked with Kyle Franke)}

\maketitle

\newtheorem*{prop6.31}{Proposition 6.31}

\begin{prop6.31}
\end{prop6.31}
	Let p be prime and m,n $\in \N$. If $p | mn$ then
$p | m$ or $p | n$.
\begin{proof}
	Assume $p \nmid m$. We must prove $p \mid n$. By the definition of greatest common divisor, we have $qp + rm = gcd(p,m)$. Because p is prime and $p \nmid m$, we know gcd(p,m) = 1. Thus we have $qp + rm = 1$. If we multiply by $n$ on both sides, we have $(qp + rm)n = qpn + rmn = n$. Because $p | mn$ and $p | qpn$, $p | (qpn + rmn)$. Thus, $p | n$.
\end{proof}

\newtheorem*{thm6.35}{Theorem 6.35}
\begin{thm6.35}
	If $m \in \Z$ and p is prime, then \center $m^{p} \equiv m$ (mod p).
\end{thm6.35}

\begin{proof}
	\textbf{Case 1.} m = 0. $0^{p} \equiv 0$. Clearly, $0^{p} \equiv 0$ (mod p).
	\\\indent\textbf{Case 2.} $m \geq 1$
	We argue by induction on $m \geq 1$ for P(m) "$m^{p} \equiv m$ (mod p)."
	\\\textbf{Base.} m = 1. So $1^{p} = 1$. Clearly, $1^{p} \equiv 1$ (mod p).
	\\\textbf{Successor.} Suppose P(m) holds. Consider $(m + 1)^{p}$. By the binomial theorem, $(m + 1)^{p} = \sum_{n=0}^{p} {{p}\choose{n}} m^{n} \cdot 1^{p-n}
	= \sum_{n=0}^{p} {{p}\choose{n}} m^{n} = {{p}\choose{0}} m^{0} + \sum_{n=1}^{p-1} {p\choose n} m^{n} + {p\choose p} m^{p}$. By proposition 6.34, $p | {p\choose n}$ and $p | {p\choose n} m^n$ for $1 \leq n \leq p-1$. Thus, $p | \sum_{n=1}^{p-1} {p\choose n} m^n$. We may write $\sum_{n=1}^{p-1} {p\choose n} m^n$ as $p \cdot j$ for some j. Hence, $(m + 1)^p = {p\choose 0} + p \cdot j + {p\choose p} m^{p} = 1 + p \cdot j + m^{p}$. We now see that $(m + 1)^p$ mod p = $1 + pj + m^{p}$ mod p = 1 mod p + pj mod p + $m^{p}$ mod p = 1 mod p + 0 mod p + $m^{p}$ mod p. = $(1 + 0)$ mod p + $m^p$ mod p = 1 mod p + $m^p$ mod p. By induction, we may rewrite this as 1 mod p + m mod p = (1 + m) mod p. We conclude that $(m + 1)^p \equiv (m + 1)$ mod p. This completes the induction.
	\\\textbf{Case 3.} $m \leq - 1$. 
	\\\textbf{Base.} m = -1. So $-1^{p} = \pm1$. Clearly, $-1^{p} \equiv -1$ (mod p).
	\\\textbf{Successor.} Suppose P(m) holds. Consider $-(m + 1)^{p}$. Because we've already proven P(m + 1) holds and $\equiv$ is an equivalence relation, we may negate both sides of the equality to see that $-(m + 1)^p \equiv -(m + 1)$ mod p
\end{proof}

\newtheorem*{prop6.33}{Proposition 6.33}
\begin{prop6.33}
	Let m, n $\in \N$. If m divides n and p is a prime factor of n that is not
	a prime factor of m, then m divides $\frac{n}{p}$ .
\end{prop6.33}

\begin{proof}
	Since $m | n$, we can find $j \in \Z$ such that $m \cdot j = n$. From Euclid's lemma, since $p | n$, it must be the case that $p | m$ or $p | j$. Since $p | j$, we can write $j = i \cdot p$, so $n = m \cdot i \cdot p$. We can conclude that $m | \frac{n}{p}$.
\end{proof}

\newtheorem*{lemma}{Lemma}
\begin{lemma}
	Let p be a prime. If $p | $($a_1 \dots a_{n}$), then $p | a_{i}$ for some $1 \leq i \leq n$.
\end{lemma}

\begin{proof}
	let P(k) be the statement "$p | a_{i}$" for some $1 \leq i \leq k$. Let's first observe P(1).
	\\\textbf{Base.} n = 1. $p | a_1$ because it is the only number given to us.
	\\\textbf{Base.} n = 2. $p | a_1 \cdot a_{2}$. By Euclid's lemma, p must divide $a_{1}$ or $a_{2}$.
	\\\textbf{Successor.} Assume P(n) holds. That is, $p | a_{i}$ for some $1 \leq i \leq n$. Consider the event that $p | $($a_1 \dots a_{n+1}$). By Euclid's lemma, either $p | $($a_1 \dots a_{n}$) or $p | a_{n + 1}$. If $p | a_{n + 1}$, we are done. Otherwise, our induction hypothesis states that $p | a_{i}$ for some $1 \leq i \leq n$. Therefore the proposition holds by induction.
\end{proof}

\end{document}
